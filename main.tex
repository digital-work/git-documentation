\documentclass{article}
\usepackage[utf8]{inputenc}

\title{Git Documentation}
\author{Joschua Thomas Simon-Liedtke (oleaf)}
\date{July 2019}

\begin{document}

\maketitle

\section{Introduction}
This document contains some helpful commands that can be used in git.

\section{Setup}

\subsection{General settings}

\subsubsection{Changing the user and their options}

On your PC/Mac:
\begin{verbatim}
	git config --global user.email "email@example.com" \newline
	got config --global user.name "John Doe"
\end{verbatim}
You can check that everything has been set correctly by typing
\begin{verbatim}
	got config --global user.name
\end{verbatim}
\\

On Overleaf:
Go to Account\textgreater  Account Settings and add the email address. Confirm the address by opening the email that you received in your mailbox.
\\

On GitHub:
Go to Settings\textgreater  Emails and add the email address. Confirm the address by opening the email that you received in your mailbox. Choose a primary email address(, which is the one that is going to show up in your commits). OBS: Make sure that you did not mark the "Keep my email address private"-option. The name (that will actually show up in your commits) can be changed in your GitHub profile.


\subsection{Cross-platform text documents}
In case you want to version control a text document with GitHub, follow this tutorial.

\subsubsection{Starting in Overleaf}

\begin{enumerate}
    \item Make a new project in Overleaf.
    \item Choose the GitHub option in the menu and choose "Create a GitHub repository". Choose a name and add a description if wanted.
    \item Choose the GitHub option in the menu and choose "Push Overleaf changes to GitHub".
    \item Switch to your GitHub account and check if everything showed up correctly.
    \item Open a terminal on your computer, create and navigate to the folder in which you want to clone the repository.
    \item Obtain the url to the repository on the GitHub page under "Clone or download" and write the following lines in the terminal: 
    \begin{verbatim}
    git clone <path/to/github/repository> <folder>
    \end{verbatim}
    (If no folder is specified, it will take the repository name as default name.)
\end{enumerate}

\subsubsection{Starting on GitHub}

\subsubsection{Starting on your PC/Mac}

\section{Maintaining the code}

Before committing new code, always make sure to update your local version from origin first!

\subsubsection{From Overleaf}

\begin{enumerate}
    \item Make your changes and save with Ctrl + s / Cmd + s.
    \item Update your Overleaf repository from origin by choosing the GitHuboption in the menu and choosing "Pull GitHub changes into Overleaf".
    \item Choose the GitHub option in the menue and choose "Push Overleaf changes to GitHub". Write a comment and commit.
\end{enumerate}

\subsubsection{From your PC/Mac}
All commands can be done from the command line after navigating to the folder containing the git repository:
\begin{verbatim}
cd <path\to\github\repository>
\end{verbatim}

\begin{enumerate}
    \item Make your changes and save them.
    \item Update your local repository from origin:
	\begin{verbatim}
    git pull   
    \end{verbatim}
	\item Stage the file(s) that have changes.
	\begin{verbatim}
    git add <name-of-file>
    \end{verbatim}
	\item Commit the changes to the local repository with a change comment:
	\begin{verbatim}
    git commit -m "Some useful and readable comment explaining the changes that you have made."
    \end{verbatim}
	\item Push the changes to origin.
	\begin{verbatim}
    git push
    \end{verbatim}
\end{enumerate}

\section{Branches}

\subsection{General commands}
Making new branch:
\begin{verbatim}
    git branch <branch-name>
\end{verbatim}

\begin{verbatim}
    git branch -d <branch-name>
\end{verbatim}

\subsection{Recurring bugs}

"Git push master fatal: You are not currently on a branch"
Explanation on https://stackoverflow.com/questions/30471557/git-push-master-fatal-you-are-not-currently-on-a-branch/30471627 or https://stackoverflow.com/questions/4735556/git-not-currently-on-any-branch-is-there-an-easy-way-to-get-back-on-a-branch.
General idea:
\begin{verbatim}
    git branch <tmp-branch>
	git checkout master
	git merge <tmp-branch>
	git push origin master
\end{verbatim}




\end{document}
